%!TEX root = ../main.tex
\section{\isklearn: a fully-functional AutoML system}
\label{sec:isklearn}

As previously discussed, an AutoML approach based on algorithm configuration comprises (i)~a configuration space, including a meta-description of a portfolio and valid domains for the hyperparameters of the algorithms that comprise it, and; (ii)~an experimental setup to evaluate candidates, enabling the AutoML approach to select/configure an algorithm that is high-performing for the input dataset. In this section, we propose a configuration space and setup, which we will assess on a set of relevant ML datasets in Section~\ref{sec:results}.

%!TEX root = ../main.tex
\subsection{Configuration Space}

\begin{lstlisting}[caption={\small \smallisklearn template described as a grammar.}, label={alg:grammar},captionpos=b, float, basicstyle=\footnotesize\ttfamily]]
S -> Preprocessing Prediction
Preprocessing -> Scaling FE | none
Scaling -> True | False
FE -> Selection | Selection Extraction 
	| Extraction Selection | Extraction
Prediction -> Scaling Predictor
\end{lstlisting}

The configuration space proposed in this work is modeled as a template, given in Algorithm~\ref{alg:grammar}. The template models a standard ML pipeline architecture, comprising two high-level components. The first, \texttt{\small Preprocessing}, represents a feature preprocessing stage, performed over the data prior to fitting the model. In contrast to Auto-WEKA and \autosklearn, our template offers a single choice of data preparation, namely \texttt{\small Scaling} through standardization. Our rationale for this vanilla version is that data preparation is a fairly important part of the data science process, and that attempting to automatically engineer the whole process would exceed the scope of this work. For this reason, the datasets we later adopt for evaluating pipelines are subject to manual preparation, as we will detail in supplementary material\footnote{\url{https://github.com/carlosemv/irace-automl-cec2021}}. 

%he pre-scaling step may be selected if there are any feature engineering components being considered. Pre-scaling, as well as scaling, consists of removing the mean (except for sparse datasets) and scaling to unit variance, using the StandardScaler class. The scaling step, as opposed to the pre-scaling one, is always an option for the configurator.

Besides \texttt{\small Scaling}, the \texttt{\small Preprocessing} component comprises feature engineering~(\texttt{\small FE}). Available options are feature \texttt{\small Selection} and \texttt{\small Extraction}, which can be used simultaneously and, if so, in any order. We provide these possibilities as a manual pipeline design typically selects between these choices, but an automated design might benefit from using both. Algorithmic options for these components  are given in Table~\ref{tb:components} and further detailed in the supplementary material.%, for brevity. 

In the case of feature \texttt{\small Extraction}, options are dimensionality reduction algorithms that vary depending on the characteristics of the dataset provided. 
%: for sparse datasets, \textit{truncated singular value decomposition}~(SVD); otherwise, \textit{principal component analysis}~(PCA), \textit{independent component analysis}~(ICA), or \textit{dictionary learning}~(DL).
%
Options for component \texttt{\small Selection} are organized into groups, namely \emph{univariate} and \emph{multivariate}. Univariate feature selection retrieves a certain percentile of features based on a given scoring function computed between each feature and the target variable. \isklearn provides the most common functions available in scikit-learn, detailed in the supplementary material for brevity. Conversely, multivariate selection fits a feature importance model using a predictor, and retrieves only the most relevant. Table~\ref{tb:components} lists the predictors available for multivariate selection, which we choose due to their balance between efficacy and efficiency when used with their suggested default parameters.
%, \isklearn provides a choice among several prediction algorithms.
%: \emph{decision trees}~(DT,~\cite{decisiontrees}), \emph{random forests}~(RF,~\cite{randomforests}), and \emph{support vector machines}~(SVM), for either classification or regression, and; \emph{linear regression}~(LR), for regression only.%
%\footnote{When used in the context of feature selection, we adopt these algorithms with their suggested default parameters, since the configuration of nested models is a complex aspect to be addressed in future work.}
Furthermore, multivariate selection can be performed recursively, using the \emph{recursive feature elimination}~(RFE) approach.

\begin{table}[!t]
\centering
\caption{\small Algorithms considered for each template component.}
\label{tb:components}
\scalebox{0.8}{
\begin{tabular}{lp{4.3cm}p{2.5cm}}
\hline
\textbf{Component} & \textbf{Algorithms} & \textbf{Conditions} \\ \hline
\multirow{2}{*}{\texttt{Extraction}} & SVD & sparse datasets\\\cline{2-3}
				    & PCA, ICA, DL & otherwise \\ \hline
\texttt{Selection} & \emph{univariate},\linebreak \emph{multivariate}  \\ \hline
\multirow{2}{*}{\emph{multivariate}} & DT, RF, SVM & classif. \& regres.\\\cline{2-3}
		&  LR & regression\\ \hline
\multirow{2}{*}{\texttt{Predictor}} & LR, DT, RF, SVM, kNN, MLP, AB & classif. \& regres.\\\cline{2-3}
		&  LR & regression\\ \hline
\end{tabular}}
\\[1em]
(LR stands for linear or logistic regression, depending on the task.)
\end{table}

The second high-level component of \isklearn is \texttt{\small Prediction}, where model fitting is actually performed. For this component, we consider a representative subset of the estimators available in \sklearn, listed in Table~\ref{tb:components}. Our rationale with this subset is that it represents most families of relevant approaches, such as generalized linear models, trees, manifold learning, neural networks, and ensembles. Options available vary according to the task nature, as \isklearn is able to cope with both classification and regression.
%Besides the options already discussed for model-based selection~(DT, RF, SVM, and LR), we also consider \emph{k-nearest neighbors}~(kNN), \emph{multi-layer perceptron}~(MLP,~\cite{mlp}), and \emph{AdaBoost}~(AB,~\cite{adaboost}). 
Finally, being heuristic algorithms, these predictors present hyperparameters of their own, which we expose for configuration. The details on the hyperparameters exposed for each predictor and their valid domains are given as supplementary material.
%\input{sections/configuration}
%\input{sections/setup}

\subsection{Configuration setup}
\label{sec:config-setup}

As previously discussed, the most important factors comprising a configuration setup concern the~(i)~problem samples provided; (ii)~performance metric adopted, and; (iii)~resource limits allowed. Below, we discuss each of these topics, starting from problem sampling, where our contributions lie;

\textbf{Problem samples.} AutoML through algorithm configuration requires three sampling levels. The top level evaluates the generalization of the AutoML approach; following the literature, we adopt holdout for this stage~\cite{autoweka,auto-sklearn}. We refer to this split as \emph{seen} and \emph{test}, since the configurator is never provided test samples. Conversely, the bottom level sampling evaluates the generalization of a candidate configuration on a subset of problem samples. In the literature, Auto-WEKA and \autosklearn once again adopt holdout. Conversely, we generalize this sampling level and test 5-fold cross-validation instead. Though this change could increase the computational cost to train a single candidate configuration, it improves the application of the AutoML systems to time series (TS) problems. More precisely, walk-forward cross-validation is a better fit for training time series models than holdout.

Finally, the mid-level sampling provides variability to the configuration process, defining what samples are seen by the bottom level sampling when a candidate configuration must be evaluated. An extreme alternative is to provide each candidate evaluation the whole dataset, at a significant computational cost. In addition, the variability between runs from a single configuration on the whole dataset is expected to be reduced for some predictors, even if the dataset is shuffled every time. The other extreme alternative would be to provide candidate configurations as little samples as possible. %~($k$, in the case of $k$-fold cross validation). 
Though fast, \irace would hardly see enough samples to avoid overfitting. %selecting candidates that overfit.

In the literature, Auto-WEKA and \autosklearn partition \emph{seen} samples into $k$ folds for this mid-level sampling. For clarity, we will dub these folds \emph{meta-folds}. Concretely, every time the configurator must evaluate a candidate, it is provided a single meta-fold, which is then subject to the bottom level sampling. Here, we also split the dataset into $k$ meta-folds, but once again we generalize the sampling and provide the configurator $p$ meta-folds at a time. The evaluation is performed using the bottom sampling on each of the $p$ meta-folds independently, and results are averaged by the configurator. We remark that both our mid-level and bottom level sampling methods may be more expensive than those used by Auto-WEKA and \autosklearn. Our goal is to provide more samples to the training phase of each individual model, particularly when the dataset considered is not significantly large or there exists some level of class imbalance. Further discussion on how to set $k$ and $p$ is provided as supplementary material.

\textbf{Performance metric.} The choice of performance metric is more related to the problem one wants to address than to the nature of the configurator considered. For regression, typical choices include $R^2$ or (R)MSE. For classification, configuration for balanced datasets may adopt the traditional accuracy metric, whereas configuration for unbalanced datasets may benefit more from Matthews correlation coefficient~(MCC).

\textbf{Resource limits.} \isklearn does not present a priori concerns with memory resources. By contrast, time is a critical factor in two major aspects. The first is the budget provided to \isklearn, which \irace uses to compute the maximum number of iterations and the number of candidate configurations to be sampled at each iteration. Also, one must set a cutoff time for the evaluation of a given candidate, or else a very expensive model fitting may compromise the configuration.

%Altogether, the In the next section, we assess the proposals discussed above.
%260 --dataset natal-crimes --task regression --sparse False --f_eng1 Selection --f_eng2 None --pre_scaling False --selection SelectFromModel --sel_model SVM --sel_threshold median --scaling False --algorithm LinearRegression
%
%318 --dataset boston-crimes --task regression --sparse False --f_eng1 Selection --f_eng2 Extraction --pre_scaling True --extraction FastICA --ext_components 0.5779 --ica_algorithm deflation --ica_fun exp --selection SelectFromModel --sel_model SVM --sel_threshold median --scaling False --algorithm LinearRegression

\medskip

Altogether, the configuration space and setup proposed in this section render \isklearn a fully-functional AutoML system, which we evaluate in the next section.